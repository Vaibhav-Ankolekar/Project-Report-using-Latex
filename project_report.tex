\documentclass[12pt,a4paper]{report}
\usepackage{graphicx}
\usepackage{amsmath}
\usepackage{fancyhdr}
\usepackage{cite}
\usepackage{framed}
\usepackage{a4wide}
\usepackage{float}
\usepackage{epsfig}
\usepackage{longtable}
\usepackage{enumerate}
\usepackage{afterpage}
\usepackage{multirow}
\usepackage{ragged2e}
\usepackage{gensymb}
\usepackage{amsfonts} 
\usepackage[left=3.5cm,top=1.5cm,right=3cm,bottom=4cm]{geometry}
\usepackage{setspace}           
\usepackage{float}
\usepackage{txfonts}
\usepackage{lipsum}

\newcommand{\Usefont}[1]{\fontfamily{#1}\selectfont}

\usepackage{lscape} % for landscape tables
\renewcommand{\baselinestretch}{1.7} 

\usepackage{blindtext}
\usepackage{xpatch}
\usepackage{url}
\usepackage{leqno}
\usepackage{subcaption}

\linespread{1.5}
\usepackage[intoc, english]{nomencl}
\hyphenpenalty=5000
\tolerance=1000
\usepackage[nottoc]{tocbibind}
\bibliographystyle{IEEEtran}
\renewcommand{\bibname}{References}

% ************************* Figures ********************************
% Save all figures in the folder figures and include them in your 
% report using the command \includegraphics{figure-name}

\graphicspath{{figures/}}

% figure files can be in jpeg,jpg, png or pdf formats
% ******************************************************************


\begin{document}
	
	
% **** The entries in this section are to be filled in by the student with appropriate values *************

% These values are used thoroughout the report 
% please fill in the appropriate values in the brackets {}

\gdef \title{Technical Project Report Template using \LaTeX } % Project title
%\gdef \author{Student Name}	 %student name
\gdef \dept{Information Technology} %Department
\gdef \degree{Bachelor of Engineering} %degree
\gdef \branch{Information Technology} %branch

\gdef \college{Bharati Vidyapeeth College of Engineeing} % Name of the College
\gdef \collegeplace{Navi Mumbai} % Location of the College

\gdef \studentA{Bhoir Ankit Bharat} %Project batch member 1
\gdef \studentAroll{4407} % Project batch member 1 ktu id

\gdef \studentB{Chouhan Aman Ravi} %Project batch member 2
\gdef \studentBroll{4414} % Project batch member 2 ktu id

\gdef \studentC{Badge Aman Ramesh} %Project batch member 3
\gdef \studentCroll{4404} % Project batch member 3 ktu id

\gdef \studentD{Ankolekar Vaibhav Pandurang} %Project batch member 4
\gdef \studentDroll{4401} % Project batch member 4 ktu id

\gdef \guide{Prof. Project guide} %Project guide
\gdef \guidedes{Assistant Professor}%project guide designation

\gdef \guideco{Prof. Project coguide} %Project coguide
\gdef \guidecodes{Assistant Professor}%project coguide designation

\gdef \guideext{Prof. Project ext guide} %Project external organisation guide
\gdef \guideextdes{Engineer/Scientist}%project external guide designation
\gdef \guideextorg{External guide organization} % Project external guide organization

\gdef \projcordinatorA{Prof. Project coordinator}% Project coordinator 1 
\gdef \projcordinatorAdes{Assistant Professor}% Project coordinator 1 designation

\gdef \projcordinatorB{Prof. Project coordinator 2} % Project coordinator 2 
\gdef \projcordinatorBdes{Assistant Professor}% Seminar coordinator 2 designation

\gdef \hod{Dr. Shankar Patil} %Head of Department
\gdef \hoddes{Professor and HOD} %HOD designation

\gdef \acadyear{2022 - 23} % Academic year
\gdef \month{September 2022} %Month of Report submission
\gdef \date{24-09-2022} %Date of signing the declaration

%*******************************************************************
% The font pages. The source tex files are there in the folder
%==================================coverpage.tex================================


\newenvironment{coverpage}
\thispagestyle{empty}
\begin{titlepage}
	\begin{center}
		{\Usefont{phv} \Large \bf \title \par}
		\vspace*{40pt}
		\large \em \Usefont{pzc}{ 
			A Project Report \par
			Submitted to the Bharati Vidyapeeth College of Enigneering\\
			in partial fulfillment of requirements for the award of degree}\\ [.15\baselineskip] \par
		\Usefont{ppl} {\bfseries  \degree}\\
		in\\
		{\Usefont{ppl} {\bfseries \branch}}\\
		by\\
		\bf {\studentA} (\studentAroll)\\
		\bf {\studentB} (\studentBroll)\\
		\bf {\studentC} (\studentCroll)\\
		\bf {\studentD} (\studentDroll)\\
		\vspace*{40pt}
		\centering
		\begin{figure}[h!]
			\centerline{\includegraphics[scale=0.5]{cet_logo}}
		\end{figure}
		
		\vspace{\stretch{0.5}}
		\footnotesize{\bf DEPARTMENT OF INFORMATION TECHNOLOGY} \par
		\bf{BHARATI VIDYAPEETH COLLEGE OF ENGINEERING} \par
		\bf{NAVI MUMBAI} \par
		\bf{\month}
	\end{center}		
\end{titlepage}	
 
% Unless essential Do not edit this tex file

%%********************Certificate*******************

% To print name of only the project coordinator 1 in the certificate page
% ==================================certificate1.tex================================
% To print name of only the seminar coordinator 1 in the certificate page

\newenvironment{certificate1}
\newpage

\begin{center}	
\textbf{DEPARTMENT OF INFORMATION TECHNOLOGY}	
\textbf{BHARATI VIDYAPEETH COLLEGE OF ENGINEERING}	
\textbf{\acadyear} 
\end{center}

\begin{center}
	\includegraphics[scale=0.4]{cet_logo}	
\end{center}
\begin{center}
	\textbf{CERTIFICATE}
\end{center}

This is to certify that the report entitled \textbf{\title} submitted by \textbf{\studentA}\hspace*{2pt}(\studentAroll),\hspace*{2pt}\textbf{\studentB}\hspace*{2pt}(\studentBroll),\hspace*{2pt}\textbf{\studentC}\hspace*{2pt}(\studentCroll) and \textbf{\studentD}\hspace*{2pt}(\studentDroll) to the Bharati Vidyapeeth College of Engineering in fulfillment of the B.E.\ degree in \branch \hspace*{2pt} is a bonafide record of the project work carried out by him under our guidance and supervision. This report in any form has not been submitted to any other University or Institute for any purpose.


\begin{singlespace}
	\vspace*{3cm}
	\begin{table}[h!]
		\centering
		\begin{tabular}{c p{3cm} c} 
			\textbf{\guide} & & \textbf{\projcordinatorA} \\
			(Project Guide) & &  (Project Coordinator)\\
			Dept.of IT & & Dept.of IT\\
			Bharati Vidyapeeth College & & Bharati Vidyapeeth College \\
			of Engineering, Navi Mumbai & & of Engineering, Navi Mumbai\\
		\end{tabular}
		
	\end{table}
	
	\vspace*{2cm}
	
	\begin{center}
		
		%\hline
		\textbf{\hod} \\ 
		\hoddes\\ 
		Dept.of IT\\ 
		College of Engineering\\
		Navi Mumbai\\
		
	\end{center}
\end{singlespace}

\thispagestyle{empty}



 

% To print names of both the project coordinators in the certificate page
% \include{certificate2}

%%***************************************************


\include{declaration} %Unless essential Do not edit this tex file

\pagenumbering{roman} 

%%********************************Abstract***********************
\include{abstract} % Please type in the abstract in this tex file abstract.tex

%%***************************************************
% Default Acknowledgement page
\include{acknowledgement}  %Unless essential Do not edit this tex file


%%***************************************************
%%**If you have only one seminar coordinator faculty member
% please comment the above line and uncomment this line

%\include{acknowledgement1}  %Unless essential Do not edit this tex file
%*******************************************************************

\thispagestyle{empty}
\newpage
    
%%**********************Table of Contents***********************
\tableofcontents
\listoffigures
\listoftables
\include{symbol} %List of Symbols (Optional) comment if not required.
% symbold may be added in the file symbol.tex

%%********************Body of the report**********
% Arabic numbering is used in the body of the report

\cleardoublepage
\setcounter{page}{1}
\pagenumbering{arabic}

%%********************Chapter 1**********
\chapter{Introduction}
\lipsum[1] % Please comment this line and type in the introduction chapter

%%********************Chapter 2**********
\chapter{Literature Review}
Each chapter is to begin with a brief introduction (in 4 or 5 sentences) about its contents. The contents can then be presented below organised into sections and subsections.

Technical writing is writing or drafting technical communication used in technical and occupational fields\cite{india}, such as computer hardware and software\cite{rpi}, engineering, chemistry, aeronautics, robotics, finance\cite{japan}, medical, consumer electronics, biotechnology, and forestry. Technical writing encompasses the largest sub-field in technical communication. See figure \ref{net2} that shows the autonomous systems in Internet.

\begin{figure}[h!]
	\centering
	\includegraphics[width=0.9\linewidth]{ospf}
	\caption{Autonomous System Hierarchy}
	\label{net2}
\end{figure}

\section{section1}
\lipsum[2] % Please comment this line and type in the content


\subsection{title 2}
\lipsum[3] % Please comment this line and type in the content

\noindent The system is described by the equation \ref{sys_eq1} below. Here y is the ordinate and x is the abscissa , m is the slope and c a constant.

\begin{equation} \label{sys_eq1}
y = mx + c
\end{equation}
\noindent Page centered and unnumbered multiple equations. The * symbol supresses equation numbering.
% Page centered and unnumbered equations
\begin{align*}
2x - 5y &=  8 \\ 
3x + 9y &=  -12
\end{align*}

\noindent Side by side figures can be created using this environment. See fig \ref{wave} below.
\begin{figure}[h!]
	\centering
	\begin{subfigure}[b]{0.4\textwidth}
		\includegraphics[width=\textwidth]{sinewave}
		\caption{Sine Wave}
		\label{fig:1}
	\end{subfigure}
	\hspace{20mm}
	\begin{subfigure}[b]{0.4\textwidth}
		\includegraphics[width=\linewidth]{cosine}
		\caption{Cosine Wave}
		\label{fig:2}
	\end{subfigure}
\caption{The Sine and Cosine waves}
\label{wave}
\end{figure}

%%********************Chapter 3**********
\chapter{System Development}
Each chapter is to begin with a brief introduction (in 4 or 5 sentences) about its contents. The contents can then be presented below organised into sections and subsections.

\section{section1}
\lipsum[2] % Please comment this line and type in the content


\subsection{title 2}
\lipsum[3] % Please comment this line and type in the content

%%********************Chapter 4**********
\chapter{Results and Discussion}
Each chapter is to begin with a brief introduction (in 4 or 5 sentences) about its contents. The contents can then be presented below organised into sections and subsections.

\section{section1}
\lipsum[2] % Please comment this line and type in the content


\subsection{title 2}
\lipsum[3] % Please comment this line and type in the content
\begin{table}[h!]
	\centering
	\caption{test table}
	\vspace*{5pt}
	\begin{tabular}{|c|c|c|}
		\hline
		Sl. No & Item 1 & Itm 2 \\ \hline
		1      & 37     & 45    \\ \hline
		2      & 42     & 23    \\ \hline
		3      & 47     & 1     \\ \hline
		4      & 52     & -21   \\ \hline
		5      & 57     & -43   \\ \hline
		6      & 62     & -65   \\ \hline
		7      & 67     & -87   \\ \hline
		8      & 72     & -109  \\ \hline
		9      & 77     & -131  \\ \hline
		10     & 82     & -153  \\ \hline
	\end{tabular}
\end{table}

%%********************Chapter 5**********
\chapter{Conclusion}
Each chapter is to begin with a brief introduction (in 4 or 5 sentences) about its contents. The contents can then be presented below organised into sections and subsections.

\lipsum[2] 

%%********************References**********
%%****This template uses IEEE bibliography style

 \begin{thebibliography}{99}
	\bibitem{india} HU, Yun Chao, et al., \emph{Mobile edge computing?A key technology
		towards 5G}, ETSI white paper, 2015, vol. 11, no 11, p. 1-16.
	
	
	\bibitem{rpi}
	@online{ Raspberry pi,
		\url{https://www.raspberrypi.org/}
		Online; accessed 10-June-2019
	}
	
	\bibitem{japan} HU, Yun Chao, et al., \emph{Mobile edge computing?A key technology
		towards 5G}, ETSI white paper, 2015, vol. 11, no 11, p. 1-16.		
\end{thebibliography}

\end{document}